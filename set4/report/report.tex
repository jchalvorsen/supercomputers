\documentclass{article}
\usepackage{amsmath}
\usepackage{bm}
\usepackage[utf8]{inputenc}
\usepackage{graphicx}
\usepackage{caption}
\usepackage{subcaption}
\usepackage{cite}
\usepackage{dsfont}
\usepackage{amssymb}
\title{TMA4280}
\author{Authors}

% Useful commands


\begin{document}
\maketitle

\begin{abstract}
Abstract. 
\end{abstract}

\newpage
\section*{Something}
In this project we have studied the vector $v \in \mathbb{R}^n$ where the elements are defined as 
\begin{equation}
\label{vdefinition}
	v(i) =\frac{1}{i^2},\quad i = 1,2,...,n.
\end{equation}
Generating the vector $v$ requires $3n$ floating point operations. For each element in the vector we do one multiplication, one division and one assignment.  
The sum $S_n$ of all the vector elements 
\begin{equation}
\label{Sdefinition}
	S_n = \sum_{i=1}^n v(i)
\end{equation}
requires $n+1$ floating point operations, $n$ for summing all the elements and one for assigning the sum.

First, a single processor C program (sum.c) generating $v$, computing $S_n$ in double precision and computing and printing the difference $S - S_n$ for different values of $n$ was written. 
\\\\
Next, the program was changed to utilize shared memorize paralellization through OpenMP. 
\\\\
Then, a program computing the sum using $P	$ processors and a ditributed memory model, as described in (REFERANSE), was made. In this program we used the MPI calls MPI\_ SCATTER and MPI\_ GATHER. 
\end{document}
