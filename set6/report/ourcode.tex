Converting the serial code in algorithm \ref{code:serial} to parallel is split in two parts, one part for using MPI and one part for using openMP. Since the MPI code needs most drastic change we start with that.

\subsection*{MPI parallelization}
Inspection of the code in algorithm \ref{code:serial} reveals the following tricks that enable us to write parallel code from the start to the end:
\begin{itemize}
\item Each MPI processor can compute the needed part from $\textbf{B}$.
\item The whole $\lambda$ vector is needed in each MPI processor given that we split $\textbf{B}$ column or row wise.
\item Given the above two points, the only time we need to exchange data from MPI processor to MPI processor is in the transpose function (and when we gather the results).
\end{itemize}
Obviously, the generation of the load matrix $\mathbf{B}$ could have been done on MPI processor with rank $0$ and then distributed to the other MPI processors. However, we do not think this would have been a good choice as it is just as easy for each MPI processor to generate its own part. By doing this we distribute the workload among all the MPI processors, and we avoid some idle time waiting for information from MPI processor with rank $0$. Since the number of internal nodes are equal to $2^k-1$ (an odd number) we cant guarantee a load balanced program (given we split the $2^k-1$ columns to the processors), but we can get decently close.\\
\\
The new pseudocode for our parallel code then becomes:\\
\begin{algorithm}[H]
 \KwData{$f(x,y)$ The load function.}
 \KwData{$\mathbf{\widetilde{X}}$, buffer to store intermediate results.}
 \KwData{$\mathbf{S}$ the discrete fourier transform operator.}
 \KwResult{$\mathbf{U}$, solution matrix. }
 Generate submatrix of $\textbf{B}$ for this MPI processor using the function $f(x,y)$\;
 Generate the whole vector $\lambda$ \;
 Let $\mathbf{\widetilde{X}} = \mathbf{SB}$  \;
 Let $\mathbf{\widetilde{X}}^T = $parallel\_transpose($\mathbf{\widetilde{X}}$)   \;
 Let $\mathbf{\widetilde{B}}^T = \mathbf{S}^{-1}(\mathbf{\widetilde{X}}^T)$ \;
 Solve system (\ref{system2}): $\tilde{u}^T_{i,j} = \frac{\tilde{b}^T_{i,j}}{\lambda_i + \lambda_j} 1 \leq i, j \leq N-1$\;
 Let $\mathbf{\widetilde{X}}^T = \mathbf{S}\mathbf{\tilde{U}}^T$  \;
 Let $\mathbf{\widetilde{X}} = $parallel\_transpose($\mathbf{\widetilde{X}}^T$)   \;
 Let $\mathbf{U} = \mathbf{S}^{-1}(\mathbf{\widetilde{X}}) $ \;
 Gather the results from all MPI processors into the big $\mathbf{U}$.
 \caption{Pseudocode for parallel poisson solver using discrete sine transform.}
 \label{code:parallel}
\end{algorithm}
\noindent with the parallel\_transpose function as follows.\\
\begin{algorithm}[H]
 \caption{Parallel\_transpose function. Notice that it returns a matrix of the same dimension as its input.}
 \KwData{Matrix $\mathbf{X}$ with $n$ rows and $m \leq n$ columns.}
 \KwResult{Matrix $\mathbf{Z}$ with $n$ rows and $m \leq n$ columns.}
 Determine which elements in $\mathbf{X}$ should go which MPI processors and order linearly in memory \;
 Spread out all elements to the correct MPI processors \;
 Get new elements from other MPI processors \;
 Order the elements correctly in memory according to a transpose operation and return as $\mathbf{Z}$.
 \label{code:transpose}
\end{algorithm}
\noindent We split the load matrix column wise and notices that the parallel\_transpose function preserves the nice data structure on each MPI processor. We chose to implement the sending and receiving of data in the transpose function as an MPI\_Alltoallv call after laying out the elements in a correct linear layout first. Unpacking the data after the MPI call is just using the same structure to get it back, only remembering that we need to transpose the data.
\\ \\
Furthermore, the gathering of data into one big $\mathbf{U}$ is not really necessary. We could either use MPI\_IO to write to disk or we're not interested in the matrix at all, say you're interested in the error or the computation time we can just return that instead. All in all, this should be a really effective parallelization of the serial code since we severely limit the amount of network operations we have to do (only two MPI\_Alltoallv and one MPI\_Gather if we need result).
\subsection*{OpenMP parallelization}
The openMP parallelization was done by putting openMP pragmas at the computer intensive loops. A typical pragma could be "\#pragma omp parallel". We set schedule(static) since we know all functions should be load balanced. We make sure to make any needed variables private or shared. Especially on the operations where we need temporary buffers we must make sure that the different threads use different buffers or we are in deep shit. We choose to use openMP in almost all places where we have loops. So where we generate $\mathbf{B}$ and $\lambda$, calculate $\mathbf{SB}$ or $\mathbf{SX}$ in any form, both times we transpose $\mathbf{\widetilde{X}}$ and when we calculate $\widetilde{u}$. We tested most of these combinations and found that they all provided some small speedboosts. The only place we don't use openMP where we have a loop is where we calculate the max error. We don't use it here because  to calculate the max error we need to know the results of the earlier calculations in the loop.