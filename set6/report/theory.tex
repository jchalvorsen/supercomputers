In this project we are studying the two-dimensional Poisson problem
\begin{align*}
-\nabla^2u&=f \quad\quad\text{ in } \Omega=(0,1)\times(0,1)\\
u&=0 \quad\quad\text{ on } \partial \Omega, \nonumber
\end{align*}
where $f$ is a given load function and $u$ is the solution. We have been using two different load functions
\begin{align*}
f(x,y)&=1
\end{align*}
and
\begin{align}
\label{loadfunc2}
f(x,y)&=5\pi^2\sin(2\pi x)\sin(\pi y) \quad\text{with exact solution} \\
u(x,y)&=\sin(2\pi x)\sin(\pi y).\nonumber
\end{align}
We discretize the Laplace operator with the five-point stencil and use regular finite difference grid with $(N+1)$ points. Therefore the spacing is $h = \frac{1}{N}$ in each direction and, at first, we use a global numbering scheme. This results in a SPD algebraic system
\begin{equation}
	\mathbf{Au} = \mathbf{b}
	\label{system1}
\end{equation}
where $\mathbf{b}$ is found from the loading function times the term $h^2$. $\mathbf{A}$ is a tensor product operator $\mathbf{A} = \mathbf{I}\otimes\mathbf{T} + \mathbf{T}\otimes\mathbf{I}$ where $\mathbf{T}$ is the matrix resulting from applying the three point stencil to the one dimensional Poisson problem. Thus, using a local numbering scheme, \eqref{system1} can be stated as 
\begin{equation*}
	\mathbf{TU} + \mathbf{UT} = \mathbf{B}
\end{equation*}
We know that $\mathbf{T}$ is SPD, thus we can perform an eigendecomposition $\mathbf{T} = \mathbf{Q\Lambda Q^T}$. Now, letting $\mathbf{\widetilde{B}} = \mathbf{Q^TBQ}$ and $\mathbf{\widetilde{U}} = \mathbf{Q^TUQ}$, we finally arrive at the system 
\begin{equation}
	\mathbf{\Lambda\widetilde{U}} + \mathbf{\widetilde{U}\Lambda} = \mathbf{\widetilde{B}}.
	\label{system2}
\end{equation}
Finding $\mathbf{\widetilde{U}}$ from \eqref{system2} is a simple calculation when we know the eigenvalues of $\mathbf{T}$, and we can from this get the final answer $\mathbf{U}$. Now we utilize what we know about the Poisson problem to avoid the two matrix-matrix products needed to compute both $\mathbf{\widetilde{B}}$ and $\mathbf{U}$, which requires $\mathcal{O}(N^3)$ time. \\
\\
We utilize the fact that we know the eigenvalues $\lambda$ and eigenvectors $\mathbf{q}$ of $\mathbf{T}$, in particular the eigenvectors and the eigenvalues are
\begin{align*}
	(q_j)_i &= \sqrt{\dfrac{2}{N-1}}\sin \Big( \dfrac{ij\pi}{N-1}\Big) \\
	\lambda_i &= 2\left(1-\cos\left(\frac{i \pi}{N-1}\right) \right)\quad 1 \leq i, j \leq N-1,
\end{align*}
which lead to the following solution to system (\ref{system2}):
\begin{align*}
\tilde{u}^T_{i,j} &= \frac{\tilde{b}^T_{i,j}}{\lambda_i + \lambda_j},\quad 1 \leq i, j \leq N-1
\end{align*}
\\ \\
Notice that these eigenvectors (if normalized) are the same as the basis for the discrete sine transform (DST), given in the lecture slides. \cite{forelesning} Thus, we can use the DST to find $\mathbf{\widetilde{B}}$ and $\mathbf{U}$. Let $\mathbf{S}$ denote the DST, and $\mathbf{S}^{-1}$ be the inverse transform. We can now express $\mathbf{Q} = \sqrt{\frac{N-1}{2}}\mathbf{S}$ and $\mathbf{Q}^T = \sqrt{\frac{2}{N-1}}\mathbf{S}^{-1}$, and therefore $\mathbf{\widetilde{B}}^T = \mathbf{S}^{-1}((\mathbf{SB})^T)$ and $\mathbf{U} = \mathbf{S}^{-1}(\mathbf{S}(\mathbf{\widetilde{U}}^T))^T$. Both of these operations can be done in $\mathcal{O}(N^2\log{N})$ time. 
This leads us to the following serial code:\\

\begin{algorithm}[H]
 \KwData{$\mathbf{B}$, load matrix.}
 \KwData{$\mathbf{S}$ the discrete fourier transform operator.}
 \KwResult{$\mathbf{U}$, solution matrix. }
 Calculate $\mathbf{\widetilde{B}}^T = \mathbf{S}^{-1}((\mathbf{SB})^T)$  \;
 Solve system (\ref{system2}): $\tilde{u}^T_{i,j} = \frac{\tilde{b}^T_{i,j}}{\lambda_i + \lambda_j}, 1 \leq i, j \leq N-1$\;
 Calculate $\mathbf{U} = \mathbf{S}^{-1}(\mathbf{S}(\mathbf{\widetilde{U}}^T))^T$ \;
 \caption{Pseudocode for serial poisson solver using discrete sine transform.}
 \label{code:serial}
\end{algorithm}