In this project we are studying the two-dimensional Poisson problem
\begin{align}\\
\label{Poisson}
-\nabla^2u&=f \quad\quad\text{ in } \Omega=(0,1)\times(0,1)\\
u&=0 \quad\quad\text{ on } \partial \Omega, \nonumber
\end{align}
where $f$ is given load function and $u$ is the solution. We have been using two different load functions
\begin{align*}
f(x,y)&=1
\end{align*}
and
\begin{align}
\label{loadfunc2}
f(x,y)&=5\pi^2\sin(2\pi x)\sin(\pi y) \quad\text{with exact solution} \\
u(x,y)&=\sin(2\pi x)\sin(\pi y).\nonumber
\end{align}
We discretize the Laplace operator with the five-point stencil and use regular finite difference grid with $(n+1)$ points in each direction and use a local numbering scheme. This results in a SPD algebraic system
\begin{equation}
	\mathbf{AU} = \mathbf{B}.
	\label{system1}
\end{equation}
Since the matrix $\mathbf{A}$ resulted from applying the three point formula in $x$ and $y$, it is a tensor product operator. Thus \eqref{system1} can be stated as 
\begin{equation}
	\mathbf{TU} + \mathbf{UT} = \mathbf{B}
\end{equation}
where $\mathbf{T}$ (skrive noe om T, begrunne at den er SPD...). Since $\mathbf{T}$ is SPD, we know we can perform an eigendecomposition $\mathbf{T} = \mathbf{Q\Lambda Q^T}$. Now, letting $\mathbf{\widetilde{B}} = \mathbf{Q^TBQ^T}$ and $\mathbf{\widetilde{U}} = \mathbf{Q^TUQ}$, we finally arrive at the system 
\begin{equation}
	\mathbf{\Lambda\widetilde{U}} + \mathbf{\widetilde{U}\Lambda} = \mathbf{\widetilde{B}}.
	\label{system2}
\end{equation}
Finding $\mathbf{\widetilde{U}}$ from \eqref{system2} is a simple calculation when we know the eigenvalues of $\mathbf{T}$, and we can from this get the final answer $\mathbf{U}$. Now we utilize what we know about the Poisson problem to avoid the two matrix-matrix products needed to compute both $\mathbf{\widetilde{B}}$ and $\mathbf{U}$, which is done in $\mathcal{O}(N^3)$ time. \\
\\
We utilize the fact that we know the eigenvalues $\lambda$ and eigenvectors $\mathbf{q}$ of $\mathbf{T}$, in particular the eigenvectors are 
\begin{align*}
	(\mathbf{q}_j)_i &= \sqrt{\dfrac{2}{N}}\sin \Big( \dfrac{ij\pi}{N}\Big).
\end{align*}
Notice that these (if normalized) are the same as the basis for the discrete sine transform (Kanskje en referanse her?). Thus, we can use the DST to find $\mathbf{\widetilde{B}}$ and $\mathbf{U}$. Let $\mathbf{S}$ denote the discrete sine transform, and $\mathbf{S}^{-1}$ be the inverse transform. We can now express $\mathbf{Q} = \sqrt{\frac{N}{2}\mathbf{S}}$ and $\mathbf{Q}^T = \sqrt{\frac{2}{N}}\mathbf{S}^{-1}$, and therefore $\mathbf{\widetilde{B}} = \mathbf{S}^{-1}((\mathbf{SB})^T)$ and $\mathbf{U} = \mathbf{S}^{-1}(\mathbf{S}(\mathbf{\widetilde{U}}^T))^T$. Both of these operations can be done in $\mathcal{O}((N^2\log{N})$ time. 