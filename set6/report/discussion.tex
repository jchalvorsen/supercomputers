So far, we have only considered homogenous Dirichlet boundary conditions. In the case of non-homogenous Dirichlet boundary conditions, the load matrix $\mathbf{B}$ would be different. Most of $\mathbf{B}$ would be as before, but the first and last rows and columns should contain the boundary point information from the discretization. \\
\\
We could have implemented the boundary condition in this way: each node gets the boundary value information, and apply the boundary conditions of the rows. Since it is known how many nodes the program uses, we can tell the first and the last node separately to apply the boundary conditions for the columns(for the first node it applies the boundary condition in the first column and the last node applies the boundary conditions in the last column).