\documentclass[12pt, a4paper]{article} %Definerer skriftstørrelse, arktrype( eks a0paper, ..., a6paper, letterpaper....), dokumenttype(article, report, book, letter, beamer(presentasjon).   
\usepackage[font=small, labelfont=bf]{caption}
\usepackage{graphicx}
\usepackage{amsmath, amssymb} %  matematiske funskjoner og symboler. 
\usepackage{listings}
\usepackage{url} % urls in the bibliograpy
\usepackage[utf8]{inputenc} % 
\usepackage{float}  %
\usepackage{subcaption} % NOT compatible with the subfig package
\usepackage{authblk}
\usepackage{siunitx}
\usepackage[]{algorithm2e}



\usepackage[ 
%total={250mm,297mm},  
left=25mm,
 right=25mm,
 top=30mm,
 bottom=20mm]{geometry} % kan endre på marger 
%%%%%%%%%%%%%%%%%%%%%%%%%%%%%%%%%%%%%%%%%%%%%%%%%%%%%%%%%%%%%%%%
% Gode infosider:
% https://no.sharelatex.com/learn/Creating_a_document_in_LaTeX 
% https://no.sharelatex.com/learn/Page_size_and_margins
 
%%%%%%%%%%%%%%%%%%%%%%%%%%%%%%%%%%%%%%%%%%%%%%%%%%%%%%%%%%%%%%%%
% Det som inkluderes i \maketitle
\title{Solving the poisson problem on supercomputer \\
TMA4280: Introduction to Supercomputing}
\author[]{Jon Christian Halvorsen, Monica Kappelslåen Plassen \\and Anette Fossum Morken}
\date{}
%%%%%%%%%%%%%%%%%%%%%%%%%%%%%%%%%%%%%%%%%%%%%%%%%%%%%%%%%%%%%%%
\begin{document}
\maketitle
\RestyleAlgo{boxruled}
%\begin{abstract}

%\end{abstract}
\section*{Introduction}
In this project, we have been studying the two-dimensional Poisson problem, and a discrete solver based on diagonalization and the Discrete Sine Transform. A serial code was parallelized using MPI and OpenMP, the correctness of the solution was verified and the program tested on the cluster Kongull. 

\section*{Theory}
In this project we are studying the two-dimensional Poisson problem
\begin{align}
-\nabla^2u&=f \quad\quad\text{ in } \Omega=(0,1)\times(0,1)\\
u&=0 \quad\quad\text{ on } \partial \Omega, \nonumber
\label{Poisson}
\end{align}
where $f$ is given load function and $u$ is the solution. We have been using two different load functions
\begin{align*}
f(x,y)&=1
\end{align*}
and
\begin{align}
f(x,y)&=5\pi^2\sin(2\pi x)\sin(\pi y) \quad\text{with exact solution} \\
u(x,y)&=\sin(2\pi x)\sin(\pi y).\nonumber
\label{loadfunc2}
\end{align}
We discretize the Laplace operator with the five-point stencil and use regular finite difference grid with $(n+1)$ points in each direction and use a local numbering scheme. This results in a SPD algebraic system
\begin{equation}
	\mathbf{AU} = \mathbf{B}.
	\label{system1}
\end{equation}
Since the matrix $\mathbf{A}$ resulted from applying the three point formula in $x$ and $y$, it is a tensor product operator. Thus \eqref{system1} can be stated as 
\begin{equation}
	\mathbf{TU} + \mathbf{UT} = \mathbf{B}
\end{equation}
where $\mathbf{T}$ (skrive noe om T, begrunne at den er SPD...). Since $\mathbf{T}$ is SPD, we know we can perform an eigendecomposition $\mathbf{T} = \mathbf{Q\Lambda Q^T}$. Now, letting $\mathbf{\widetilde{B}} = \mathbf{Q^TBQ^T}$ and $\mathbf{\widetilde{U}} = \mathbf{Q^TUQ}$, we finally arrive at the system 
\begin{equation}
	\mathbf{\Lambda\widetilde{U}} + \mathbf{\widetilde{U}\Lambda} = \mathbf{\widetilde{B}}.
	\label{system2}
\end{equation}
Finding $\mathbf{\widetilde{U}}$ from \eqref{system2} is a simple calculation, and we can from this get the final answer $\mathbf{U}$. Now we utilize what we know about the Poisson problem to avoid the two matrix-matrix products needed to compute both $\mathbf{\widetilde{B}}$ and $\mathbf{U}$, which is done in $\mathcal{O}(N^3)$ time. \\
\\
(Forklare hvorfor vi kan bruke DST)
\\
\\
Let $\mathbf{S}$ denote the discrete sine transform, and $\mathbf{S}^{-1}$ be the inverse transform. We can now express $\mathbf{Q} = \sqrt{\frac{N}{2}\mathbf{S}}$ and $\mathbf{Q}^T = \sqrt{\frac{2}{N}}\mathbf{S}^{-1}$, and therefore $\mathbf{\widetilde{B}} = \mathbf{S}^{-1}((\mathbf{SB})^T)$ and $\mathbf{U} = \mathbf{S}^{-1}(\mathbf{S}(\mathbf{\widetilde{U}}^T))^T$. Both of these operations can be done in $\mathcal{O}((N^2\log{N})$ time. 


\section*{Parallelization}
Converting the serial code in algorithm \ref{code:serial} to parallel is split in two parts, one part for using MPI and one part for using openMP. Since the MPI code needs most drastic change we start with that.

\subsection*{MPI parallelization}
Inspection of the code in algorithm \ref{code:serial} reveals the following tricks that enable us to write parallel code from the start to the end: (in the following a node is referred to as one MPI processor)
\begin{itemize}
\item Each node can compute the needed part from $\textbf{B}$.
\item The whole $\lambda$ vector is needed in each node given that we split $\textbf{B}$ column or row wise.
\item Given the above two points, the only time we need to exchange data from node to node is in the transpose function (and when we gather the results).
\end{itemize}

The new pseudocode for our parallel code then becomes:

\begin{algorithm}[H]
 \KwData{$b(x,y)$ The load function.}
 \KwData{$\mathbf{\widetilde{X}}$, buffer to store intermediate results.}
 \KwData{$\mathbf{S}$ the discrete fourier transform operator.}
 \KwResult{$\mathbf{U}$, solution matrix. }
 Generate submatrix of $\textbf{B}$ for this node using the function $b(x,y)$\;
 Generate the whole vector $\lambda$ \;
 Let $\mathbf{\widetilde{X}} = \mathbf{SB}$  \;
 Let $\mathbf{\widetilde{X}}^T = $parallel\_transpose($\mathbf{\widetilde{X}}$)   \;
 Let $\mathbf{\widetilde{B}}^T = \mathbf{S}^{-1}(\mathbf{\widetilde{X}}^T)$ \;
 Solve system (\ref{system2}): $\tilde{u}^T_{i,j} = \frac{\tilde{b}^T_{i,j}}{\lambda_i + \lambda_j} 1 \leq i, j \leq N$\;
 Let $\mathbf{\widetilde{X}}^T = \mathbf{S}\mathbf{\tilde{U}}^T$  \;
 Let $\mathbf{\widetilde{X}} = $parallel\_transpose($\mathbf{\widetilde{X}}^T$)   \;
 Let $\mathbf{U} = \mathbf{S}^{-1}(\mathbf{\widetilde{X}}) $ \;
 Gather the results from all nodes into the big $\mathbf{U}$.
 \caption{Pseudocode for serial poisson solver using discrete sine transform.}
 \label{code:parallel}
\end{algorithm}

with the parallel\_transpose function as follows.

\begin{algorithm}[H]
 \caption{Parallel\_transpose function. Notice that it returns a matrix of the same dimension as its input.}
 \KwData{Matrix $\mathbf{X}$ with $n$ rows and $m \leq n$ columns.}
 \KwResult{Matrix $\mathbf{Z}$ with $n$ rows and $m \leq n$ columns.}
 Determine which elements in $\mathbf{X}$ should go which nodes and order linearly in memory \;
 Spread out all elements to the correct nodes \;
 Get new elements from other nodes \;
 Order the elements correctly in memory according to a transpose operation and return as $\mathbf{Z}$.
 \label{code:transpose}
\end{algorithm}
We split the load matrix column wise and notices that the parallel\_transpose function preserves the nice data structure on each node. We chose to implement the sending and receiving of data in the transpose function as an MPI_Alltoallv call after laying out the elements in a correct linear layout first. Unpacking the data after the MPI call is just using the same structure to get it back, only remembering that we need to transpose the data.

Furthermore, the gathering of data into one big $\mathbf{U}$ is not really necessary. We could either use MPI\_IO to write to disk or we're not interested in the matrix at all (say you're interested in the error or the computation time) we can just return that instead. All in all, this should be a really effective parallelization of the serial code since we severely limit the amount of network operations we have to do (only two MPI_Alltoallv and one MPI_Gather if we need result).


 
\section*{Hardware}
Our code is run on the Kongull which is a linux cluster using CentOS 5.3 and running Rocks. The cluster has 1 login node, 4 I/O nodes and 108 compute nodes. 96 of the compute nodes have two 6-core AMD Istanbul processors while the 12 last nodes have 2 8-core Intel Sandy Bridge processors. For more information about the cluster we refer to the clusters webpage \cite{kongull}. 
 
\section*{Convergence analysis and testing}
\begin{figure}[h]
\centering
\includegraphics[width=0.7\linewidth]{./figures/checkConv}
\caption{Shows that the result from the convergence test compared with curves with slope $h$ and $h^2$}
\label{fig:checkConv}
\end{figure}

\section*{Results}

\begin{figure}
\centering
\includegraphics[width=0.7\linewidth]{./figures/speedup}
\caption{}
\label{fig:speedup}
\end{figure}


\begin{figure}[h]
\centering
\includegraphics[width=0.7\linewidth]{./figures/checkConv}
\caption{Shows that the result from the convergence test compared with curves with slope $h$ and $h^2$}
\label{fig:checkConv}
\end{figure}

\begin{figure}
\centering
\includegraphics[width=0.7\linewidth]{./figures/runtime}
\caption{}
\label{fig:runtime}
\end{figure}

\begin{figure}
\centering
\includegraphics[width=0.7\linewidth]{./figures/efficiacy}
\caption{}
\label{fig:efficiacy}
\end{figure}


\section*{Discussion}
So far, we have only considered homogenous Dirichlet boundary conditions. In the case of non-homogenous Dirichlet boundary conditions, the load matrix $\mathbf{B}$ would be different. Most of $\mathbf{B}$ would be as before, but the first and last rows and columns should contain the boundary point information from the discretization. \\
\\
We could have implemented the boundary condition in this way: each node gets the boundary value information, and apply the boundary conditions of the rows. Since it is known how many nodes the program uses, we can tell the first and the last node separately to apply the boundary conditions for the columns(for the first node it applies the boundary condition in the first column and the last node applies the boundary conditions in the last column).

\section*{Conclusion}
We have been able to implement a parallel poisson solver using both OpenMP, MPI and a hybrid model, and verified its correctness. Using the cluster Kongull, runtime, speedup and efficiency tests were performed. The results of these tests show that the parallel code is faster than the serial implementation. Also, for our implementation, the pure distributed model outclasses the hybrid model when we limit the use of nodes. We suspect this is because alot of OpenMP calls has to be made compared to the two times data has to be sent using MPI calls. When we are liberal in the use of nodes the hybrid model seems to holding it's own ground. The best timing result we got was just above 25 seconds for the problem with $N=16384$ nodes in each spatial direction spread across 18 nodes, with $p=1$ and $t=2$ on each node for a total of 36 processors/threads. We could not get access to 36 nodes to test with $p=1$ and $t=1$. 


%%%%%%%%%%%%%%%%%%%%%%%%%%%%%%%%%%%%%%%%%%%%%%%%%%%%%%%%%
% REFERANSER
\begin{thebibliography}{10}
\bibitem{forelesning} Kvarving, Arne Morten \emph{Solving the linear system resulting from the Poisson problem.}
\bibitem{kongull} Hardware information about the Kongull cluster, gathered 26. mars 2015 from \\ \url{https://www.hpc.ntnu.no/display/hpc/Kongull+Hardware}
\end{thebibliography}
\end{document}
